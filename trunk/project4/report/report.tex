\title{CS531 Programming Assignment 4: Wumpus Agent}
\author{
        Michael Lam, Xu Hu \\
        EECS, Oregon State University\\
        %\email{}
        %\and
}
\documentclass[12pt]{article}

%\usepackage{multirow}
\usepackage[lofdepth,lotdepth]{subfig}
\usepackage{float}
\usepackage{epstopdf}
\usepackage{algorithmic}
\usepackage{algorithm}
\usepackage[english]{babel}
\usepackage{graphicx}
%\usepackage{subfig}
\usepackage{amsmath}
%\usepackage{hyperref}
%\hypersetup{
%    colorlinks,%
%    citecolor=green,%
%    filecolor=magenta,%
%    linkcolor=red,%
%    urlcolor=cyan
%}

%\underset{x}{\operatorname{argmax}}
%\underset{x}{\operatorname{argmin}}
\DeclareMathOperator*{\argmin}{arg\,min}
\DeclareMathOperator*{\argmax}{arg\,max}

\begin{document}
\maketitle

\begin{abstract}
In this assignment we design, implement and evaluate an algorithm that uses first-order logic and A* search for an agent in order to solve Wumpus puzzles.
\end{abstract}

% -------------------------------------------------
\section{Introduction}

The Wumpus world is a 4x4 grid containing pits, one Wumpus and one gold at various locations. The objective of the agent is to retrieve the gold without dying from the Wumpus or falling into a pit. Furthermore the agent can perceive its environment and infer the location of pits and the Wumpus due to the rules of constructing a Wumpus world. Therefore it makes sense to implement an algorithm involving logic and search to make intelligent decisions.

We used the existing Wumpus environment simulator for Python provided by Walker Orr. We designed an agent that uses first-order logic with a tell-ask interface to assert/query what it knows about the Wumpus world and A* search to plan routes around the Wumpus world. The algorithm is essentially the same as the one provided in the Russell-Norvig textbook (pg. 270, fig. 7.20). For answering logic queries, the program uses the Prover-9 program.

To evaluate our algorithm, we designed some experiments and collected statistics.

\section{Approaches}\label{sec:approaches}

In this section we describe our implementation of the agent and discuss our design decisions.

\subsection{Simulator}

We used the simulator for Python provided by Walker Orr. While the simulator represented the Wumpus world fairly well as described in the textbook, it was still incomplete. First of all, the simulator lacked the notion of facing directions as well as the percepts bump and scream. In addition, the simulator would declare reaching the gold a success, eliminating the need for the agent to travel back and climb at the initial square. This also differs from the textbook's formulation. Therefore, we modified the simulator to add these remaining percepts, fixed the simulator to keep track of facing direction and allowed the agent to return to the initial square to climb. The modified simulator should represent the Wumpus world as described in the textbook.

\subsection{Knowledge Base}

We designed our Knowledge Base as follows:
\begin{itemize}
	\item Atemporal
		\begin{description}
			\item[B(x,y)] Breezy at location (x,y)
			\item[P(x,y)] Pit at location (x,y)
			\item[S(x,y)] Smelly at location (x,y)
			\item[W(x,y)] Wumpus at location (x,y)
		\end{description}
	\item Temporal
		\begin{description}
			\item[WumpusAlive(t)] Wumpus is alive at time t
			\item[HaveGold(t)] Agent has gold at time t
			\item[HaveArrow(t)] Agent has arrow at time t
		\end{description}
	\item Both
		\begin{description}
			\item[Loc(x,y,t)] Agent was at location (x,y) at time t
			\item[OK(x,y,t)] Square (x,y) at time t was safe to visit
		\end{description}
	\item Perception
		\begin{description}
			\item[Breeze(t)] Breeze perceived at time t
			\item[Stench(t)] Stench perceived at time t
			\item[Glitter(t)] Glitter perceived at time t
			\item[Bump(t)] Bump perceived at time t
			\item[Scream(t)] Scream perceived at time t
		\end{description}
	\item Action
		\begin{description}
			\item[Forward(t)] Forward action at time t
			\item[TurnLeft(t)] TurnLeft action at time t
			\item[TurnRight(t)] TurnRight action at time t
			\item[Shoot(t)] Shoot action at time t
			\item[Grab(t)] Grab action at time t
			\item[Climb(t)] Climb action at time t
		\end{description}
\end{itemize}

We wrote out some rules that Prover-9 could use to prove a query. Note that in our implementation, we explicitly fill for each $x$, $y$ and $t$ instead. We know the size of the world is 4x4 so we are able to enumerate all $x$ and $y$. For each time step, we enumerate all $x$ and $y$ for time $t$. The reason for enumerating out all possibilities instead of using variables is to keep the logic relatively simple and avoid having to implement arithmetic in Prover-9.

\begin{itemize}
	\item $B(x,y) \Leftrightarrow P(x1,y1) \vee P(x2,y2) \vee P(x3,y3) \vee P(x4,y4)$
        \item $S(x,y) \Leftrightarrow W(x1,y1) \vee W(x2,y2) \vee W(x3,y3) \vee W(x4,y4)$
        \item $Loc(x,y,t) \Rightarrow (Breeze(t) \Leftrightarrow B(x,y))$
        \item $Loc(x,y,t) \Rightarrow \neg P(x,y)$
        \item $Loc(x,y,t) \Rightarrow (Stench(t) \Leftrightarrow S(x,y))$
        \item $Loc(x,y,t) \Rightarrow (\neg W(x,y)) \vee (W(x,y) \wedge \neg WumpusAlive(t))$
        \item $OK(x,y,t) \Leftrightarrow \neg P(x,y) \wedge \neg(W(x,y) \wedge WumpusAlive(t))$
\end{itemize}

Other than these fundamental rules, the rest are essentially assertions of percepts, actions and derived atemporal facts of the Wumpus world at every time step. These information should be sufficient for the agent to reason about the world and make good decisions.

\subsection{Algorithm}

We implemented the algorithm in the textbook (pg. 270, fig. 7.20). It is mostly the same in terms of the decision-making process. There are a few technical differences regarding the Knowledge Base representation.

We asserted all the facts at the beginning of the algorithm. The Make-World-Logic-Sentences function constructs sentences that assert conditional sentences of the Wumpus world at the current time. This includes sentences relating "OK" to "Pit" and "Wumpus" as well as explicitly writing out the sentences relating adjacent squares such as relating "Breeze" to "Pit." It is important to note that some of these sentences are temporal, so we create them at every new time step. This is to avoid adding arithmetic logic to Prover-9. Another note is that the adjacent squares are written out explicitly. Again this is to avoid using arithmetic as a design choice. 

One final technical note is that every call to Ask also caches the query if the query is proven true. This is to improve performance.

\begin{algorithm}[H]
\caption{Hybrid-Wumpus-Agent}
\label{hybridalg}
\begin{algorithmic}
\STATE HYBRID-WUMPUS-AGENT(percept)
\STATE $Inputs: "percept" list$
\STATE $Persistent: Knowledge Base "KB", time "t", action sequence "plan"$
\STATE $Returns: single next "action"$
\STATE Tell(KB, Make-World-Logic-Sentences(t))
\STATE Tell(KB, Make-Percept-Sentence(percept, t))
\STATE Tell(KB, Make-Location-Safe-Sentence(current))
\STATE Tell(KB, Make-Have-Arrow-Sentence())
\STATE Tell(KB, Make-Have-Gold-Sentence())
\STATE safe := \{(x,y) : Ask(KB, OK(x,y,t) = TRUE\}
\IF{Ask(KB, Glitter(t)) = TRUE}
	\STATE plan := {[}Grab{]} + Plan-Route(current, \{(0,0)\}, safe) + {[}Climb{]}
\ENDIF
\IF{plan is empty}
	\STATE unvisited := \{(x,y) : ASK(KB, Loc(x,y,t')) = FALSE for all t' $<=$ t\}
	\STATE plan := Plan-Route(current, unvisited $\cap$ safe, safe)
\ENDIF
\IF{plan is empty AND Ask(KB, HaveArrow(t)) = TRUE}
	\STATE possible\_wumpus := \{(x,y) : Ask(KB, -W(x,y)) = FALSE\}
	\STATE plan := Plan-Shot(current, possible\_wumpus, safe)
\ENDIF
\IF{plan is empty}
	\STATE not\_unsafe := \{(x,y) : Ask(KB, -OK(x,y,t)) = FALSE\}
	\STATE plan := Plan-Route(current, unvisited $\cap$ not\_unsafe, safe)
\ENDIF
\IF{plan is empty}
	\STATE plan := Plan-Route(current, \{(0,0)\}, safe) + {[}Climb{]}
\ENDIF
\STATE action := Pop(plan)
\STATE Tell(KB, Make-Action-Sentence(action, t))
\STATE t := t+1
\RETURN action
\end{algorithmic}
\end{algorithm}

\subsection{A* Search}

In the Plan-Route function we formulate an A* search problem given the agent's current location, goal locations and allowable squares. The A* algorithm is essentially the same as in the previous assignment with a different heuristic.

\begin{algorithm}[H]
\caption{A* Search}
\label{astaralg}
\begin{algorithmic}
\STATE exploredSet = $\emptyset$ 
\STATE frontier = [initialPath]
\WHILE{number(explored) $<$ NMAX}
\IF{frontier == $\emptyset$}
    \RETURN FALSE
\ENDIF
\STATE path = frontier.pop()
\STATE state = path[0]
\STATE exploredSet.add(state)
\IF{state == goalState}
    \RETURN path
\ENDIF
\FOR{action in state.validActions()}
    \FOR{newState in action.results()}
        \STATE newPath = path + newState
        \IF{ismember(frontier,newPath) == FALSE}
            \STATE frontier.push(newPath)
        \ENDIF
    \ENDFOR
\ENDFOR
\ENDWHILE
\end{algorithmic}
\end{algorithm}


\section{Experiments}\label{sec:exp}

We solved all the $77$ problems. The first figure \ref{fig:nback} shows the number of backtrackings for each problem. The ``naked triple (double)'' strategy exhibits impressive performance in constraint propagations. Without such kinds of rules, there are $1/4$ problems need to use backtracking, no matter pick the most constrained slot or pick slot randomly. Meanwhile, only $2$ or $3$ problems require backtracking for the ``naked triple'' case. 

\begin{figure}[ht]
\centering
\begin{tabular}{cc}
\subfloat[Rule 1]{\includegraphics[scale=0.32]{../results/bt-r1-bt.pdf}} 
   & \subfloat[Rule 1,2]{\includegraphics[scale=0.32]{../results/bt-r1-r2-bt.pdf}}\\
\subfloat[Rule 1 + Naked double and triple]{\includegraphics[scale=0.32]{../results/bt-r1-n23-bt.pdf}} 
   & \subfloat[Rule 1,2 + Naked double and triple]{\includegraphics[scale=0.32]{../results/bt-r1-r2-n23-bt.pdf}}\\
%\subfloat[E]{\includegraphics[width=2cm]{logo}} 
%   & \subfloat[F]{\includegraphics[width=3cm]{logo}}\\
\end{tabular}
\caption{Number of backtracing with respect to each problem.}
\label{fig:nback}
\end{figure}

Next we show the problem completion in each difficulty level in table \ref{tab:dif}. From this table, we can see that the annotation for each problem are corrected approximately, except $3$ evil problems which were solved by only simple rules. For comparing the effectiveness of most constrained slot heuristic and random slot heuristic, we make the same experiment with random slot heuristic, which placed in last 4 rows in table \ref{tab:dif}. It seems these two heuristics are getting same results. We will further do experiments for this comparison on number of rules. For justifying the difficulty of problems, the average number of filled-in numbers are concerned, as shown in table \ref{tab:fill}.
\begin{table}[!h]
    \centering
    \scalebox{0.8}{
	    \begin{tabular}{|l|c|c|c|c|c|c|c|}
		\hline
		Complexity   & r1 & r1,2 & r1,2+n2 & r1,2+n3 & r1,2+n2,3 & r1,2+bt & r1,2+n2,3+bt \\ \hline
		Easy (23):   & 21 & 21 & 23 & 23 & 23 & 23 & 23 \\ \hline
		Medimum (21) & 3 & 3 & 10 & 14 & 19 & 21 & 21 \\ \hline
		Hard (18)    & 0 & 0 & 7 & 4 & 15 & 18 & 18 \\ \hline
		Evil (15)    & 3 & 3 & 4 & 3 & 8 & 15 & 15 \\ \hline
        \hline
		Easy rand:   & 21 & 21 & 23 & 23 & 23 & 23 & 23 \\ \hline
		Medimum rand & 3 & 3 & 10 & 14 & 19 & 21 & 21 \\ \hline
		Hard rand    & 0 & 0 & 7 & 4 & 15 & 18 & 18 \\ \hline
		Evil rand    & 3 & 3 & 4 & 3 & 8 & 15 & 15 \\ \hline
	    \end{tabular}
    }
    \caption{Number of complete problems in different combinations of rules. The default heuristic is the most constrained slot. However, it seems picking most constrained slot or picking random slot is the same in this case. Here, r1: rule 1, r1,2: rule1 + rule2, n2: naked double, n3: naked triple, n2,3: n2 + n3.}\label{tab:dif}
\end{table}

\begin{table}[!h]
    \centering
    \scalebox{0.9}{
	    \begin{tabular}{|l|c|c|c|c|}
		\hline
		             & easy & medimum & hard & evil \\ \hline
		average number & 34.70 & 29.05 & 26.28 & 26.13 \\ \hline
	    \end{tabular}
    }
    \caption{Average filled-in numbers for each problem level.}\label{tab:fill}
\end{table}

The number of rules being used is an important factor, we also use it to estimate the difficulty of a problem and show the average number of different kinds of rules used for each set of problems. The results is shown in table \ref{tab:rule}. The trendency is clear that more hard problems requires more number of rules. Note that for this experiment we use the combination of all rules and backtracking for making sure every problem can be solved.

\begin{table}[!h]
    \centering
    \scalebox{0.8}{
	    \begin{tabular}{|l|c|c|c|c|c|c|c|}
		\hline
		             & r1 & r2 & n2 & n3 & bt \\ \hline
		Easy (23):   & 42.26 & 0.04 & 22.00 & 12.70 & 0 \\ \hline
		Medimum (21) & 46.95 & 0.14 & 31.14 & 16.81 & 0 \\ \hline
		Hard (18)    & 49.06 & 0.33 & 32.11 & 15.94 & 0.06 \\ \hline
		Evil (15)    & 59.40 & 0 & 38.20 & 22.33 & 0.60 \\ \hline
	    \end{tabular}
    }
    \caption{Average number of rules used by problems in different levels. Here, r1: rule 1, r2: rule2, n2: naked double, n3: naked triple, bt: number of backtrackings.}\label{tab:rule}
\end{table}

We further compare with two heuristics in the number of rules being taken. Figure \ref{fig:heu} shows all pair of comparisons. By applying these two heuristics, a problem uses more or less the same number of rules. This is consistent with our experiment in number of complete problems. 

\begin{figure}[ht]
\centering
\begin{tabular}{cc}
\subfloat[Rule 1]{\includegraphics[scale=0.32]{../results/bt-r1-r2-n23-rule1.pdf}} 
   & \subfloat[Rule 2]{\includegraphics[scale=0.32]{../results/bt-r1-r2-n23-rule2.pdf}}\\
\subfloat[Naked double]{\includegraphics[scale=0.32]{../results/bt-r1-r2-n23-naked2.pdf}} 
   & \subfloat[Naked triple]{\includegraphics[scale=0.32]{../results/bt-r1-r2-n23-naked3.pdf}}\\
%\subfloat[E]{\includegraphics[width=2cm]{logo}} 
%   & \subfloat[F]{\includegraphics[width=3cm]{logo}}\\
\end{tabular}
\caption{Comparisons between most constrained slot heuristic and random slot heuristic in number of rules taken.}
\label{fig:heu}
\end{figure}
%
%\begin{table}[!h]
%    \centering
%    \scalebox{0.9}{
%	    \begin{tabular}{|l|c|c|c|c|c|c|c|}
%		\hline
%		Disks: & 4 & 5 & 6 & 7 & 8 & 9 & 10\\ \hline
%		A*/admissible & 8.8 & 11.6 & 13.95 & 16.5 & 18.6 & n/a & n/a\\ \hline
%		%A*/admissible H7 & 8.95 & 11.6 & 13.95 & 16.75 & 19.2 & 22.4 & 24.77\\ \hline
%		RBFS/admissible & 8.8 & 11.6 & 13.95 & 16.35 & 17.11 & n/a & n/a\\ \hline
%		A*/nonadmissible 1 & 8.85 & 11.85 & 14.6 & 17.5 & 20.2 & 24.1 & 27.6\\ \hline
%		RBFS/nonadmissible 1 & 8.85 & 11.85 & 15.25 & 19 & 22.21 & 26.81 & 32.25 \\ \hline
%		A*/nonadmissible 2 & 9.25 & 12.95 & 16.5 & 20.05 & 23.8 & 27.4 & 34.05\\ \hline
%		RBFS/nonadmissible 2 & 11.25 & 17.85 & 24.35 & 32.75 & 40.35 & 50.7 & 64.55\\ \hline
%	    \end{tabular}
%    }
%    \caption{Average solution length per algorithm, heuristic and disk size. n/a means unable to compute within 10 minutes for all problems. Note that some experiments failed for completing before NMAX nodes; these were not included in the average.}\label{tab:solen}
%\end{table}

\section{Discussion}\label{sec:dis}

Discussion here.


\bibliographystyle{plain}
\bibliography{ref}

\end{document}
This is never printed
