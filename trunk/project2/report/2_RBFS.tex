\section{Recursive Best-First Search}

Recursive best-first search or RBFS works by storing an f-limit for each node. The algorithm uses the f-limit to decide which subtree of the problem tree to explore by considering the best and 2nd best (alternative) f-limits. In order to keep the search functional, RBFS also updates the f-values of each node during the search.

The advantage of RBFS over A* is that RBFS uses less memory. Whereas A* stores all of its explored nodes, RBFS will only keep relevant nodes in memory. However, the disadvantage of RBFS over A* is that RBFS could expand more nodes than A* due to redundancy. Since RBFS does not store all nodes explored, it can re-expand the same nodes and thereby increasing computation time.

The pseudocode is listed in algorithm \ref{alg2}.

\begin{algorithm}
\caption{RBFS Search}
\label{alg2}
\begin{algorithmic}
\STATE function RBFS(state, f-limit):
\IF{state is goal state}
	\RETURN solution
\ENDIF
\STATE successor = all children of state
\IF{successors is empty}
	\RETURN failure
\ELSE
	\FOR{s in successors}
		\STATE s.$f$ = $max$(s.$g$ + s.$h$, state.$f$)
	\ENDFOR
	\WHILE{\TRUE}
		\STATE best = lowest f-value node in successors
		\IF{best.$f$ $>$ f-limit}
			\RETURN failure, best.$f$
		\ENDIF
		\STATE alternative = second best f-value of any node in successors
		\STATE result, best.$f$ = RBFS(best, $min$(f-limit, alternative))
		\IF{result is not failure}
			\RETURN result
		\ENDIF
	\ENDWHILE
\ENDIF
\end{algorithmic}
\end{algorithm}

We will analyze several admissble and non-admissble heuristics in section \ref{sec:exp}.

%\begin{table}[h]
%    \centering
%    \begin{tabular}{|l|l|l|c|c|c|c|c|}
%        \hline
%        WALL & DIRT & HOME & FORWARD & RIGHT & LEFT & SUCK & OFF  \\ \hline
%        1    & 0    & 1    & 0.0     & 0.65  & 0.33 & 0.0  & 0.02 \\ 
%        1    & 1    & 0    & 0.0     & 0.0   & 0.0  & 1.0  & 0.0  \\ 
%        1    & 1    & 1    & 0.0     & 0.0   & 0.0  & 1.0  & 0.0  \\ 
%        1    & 0    & 0    & 0.0     & 0.67  & 0.33 & 0.0  & 0.0  \\ 
%        0    & 0    & 1    & 0.9     & 0.05  & 0.05 & 0.0  & 0.0  \\ 
%        0    & 1    & 0    & 0.0     & 0.0   & 0.0  & 1.0  & 0.0  \\ 
%        0    & 1    & 1    & 0.0     & 0.0   & 0.0  & 1.0  & 0.0  \\ 
%        0    & 0    & 0    & 0.8     & 0.1   & 0.1  & 0.0  & 0.0  \\
%        \hline
%    \end{tabular}
%    \caption{Parameters of multinomial distributions for each situation.}\label{tab:random}
%\end{table}
