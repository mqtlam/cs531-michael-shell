\section{Recursive Best-First Search}

Recursive best-first search or RBFS works by storing a best and alternative f-limits. The algorithm uses the f-limit to decide which subtree of the problem state to explore.

The advantage of RBFS over A* is that RBFS uses less memory. Whereas A* stores all of its explored nodes, RBFS will only keep relevant nodes in memory. However, the disadvantage of RBFS over A* is that RBFS could expand more nodes than A* due to redundancy. Since RBFS does not store all nodes explored, it can re-expand the same nodes and thereby increasing computation time.

The following is the pseudocode:

\begin{verbatim}

RBFS(state, f-limit)
    if state is the goal state 
        return solution
    successors := all children of state
    if successors is empty
        return failure
    else
        for each s in successors
            s.f := max(s.g + s.h, state f)
        loop do
            best := lowest f-value node in successors
            if best.f > f-limit
                return failure, best.f
            alternative := second best f-value of any node in successors
            result, best.f := RBFS(best, min(f-limit, alternative))
            if result is not failure
                return result
\end{verbatim}

%\begin{table}[h]
%    \centering
%    \begin{tabular}{|l|l|l|c|c|c|c|c|}
%        \hline
%        WALL & DIRT & HOME & FORWARD & RIGHT & LEFT & SUCK & OFF  \\ \hline
%        1    & 0    & 1    & 0.0     & 0.65  & 0.33 & 0.0  & 0.02 \\ 
%        1    & 1    & 0    & 0.0     & 0.0   & 0.0  & 1.0  & 0.0  \\ 
%        1    & 1    & 1    & 0.0     & 0.0   & 0.0  & 1.0  & 0.0  \\ 
%        1    & 0    & 0    & 0.0     & 0.67  & 0.33 & 0.0  & 0.0  \\ 
%        0    & 0    & 1    & 0.9     & 0.05  & 0.05 & 0.0  & 0.0  \\ 
%        0    & 1    & 0    & 0.0     & 0.0   & 0.0  & 1.0  & 0.0  \\ 
%        0    & 1    & 1    & 0.0     & 0.0   & 0.0  & 1.0  & 0.0  \\ 
%        0    & 0    & 0    & 0.8     & 0.1   & 0.1  & 0.0  & 0.0  \\
%        \hline
%    \end{tabular}
%    \caption{Parameters of multinomial distributions for each situation.}\label{tab:random}
%\end{table}
