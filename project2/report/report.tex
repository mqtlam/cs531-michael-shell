\title{CS531 Programming Assignment 2: Towers of Corvallis}
\author{
        Michael Lam, Xu Hu \\
        EECS, Oregon State University\\
        %\email{}
        %\and
}
\documentclass[12pt]{article}

%\usepackage{multirow}
\usepackage[lofdepth,lotdepth]{subfig}
\usepackage{float}
\usepackage{epstopdf}
\usepackage{algorithmic}
\usepackage{algorithm}
\usepackage[english]{babel}
\usepackage{graphicx}
%\usepackage{subfig}
\usepackage{amsmath}
%\usepackage{hyperref}
%\hypersetup{
%    colorlinks,%
%    citecolor=green,%
%    filecolor=magenta,%
%    linkcolor=red,%
%    urlcolor=cyan
%}

%\underset{x}{\operatorname{argmax}}
%\underset{x}{\operatorname{argmin}}
\DeclareMathOperator*{\argmin}{arg\,min}
\DeclareMathOperator*{\argmax}{arg\,max}

\begin{document}
\maketitle

\begin{abstract}
In this assignment we design, implement and discuss two different informed search algorithms and heuristics to solve the Towers of Corvallis, which is a variation of Towers of Hanoi. 
\end{abstract}

% -------------------------------------------------
\section{Introduction}

The Towers of Corvallis puzzle is a variation on the Towers of Hanoi puzzle. While similarly consisting of 3 pegs and $n$ disks, the Corvallis variation allows any disk to go on top of any other disk. The goal is to find the smallest number of moves in getting from an initial state to the goal state, which is defined as the order 9876543210 on peg A for 10 disks and similarly for fewer disks.

We implement two informed search algorithms: A* and RBFS (recursive best-first search). As informed searches, we also implement two heuristics, one admissible and one non-admissible. For each algorithm and heuristic function, we evaluate the performance by testing across different number of disks and different initial states.

\section{A* Search}\label{sec:astar}

We implement the A* search algorithm for finding the paths between initial states to a given goal state in the problem of Tower of Corvallis. It is summarized in algorithm \ref{alg1}.

\begin{algorithm}[H]
\caption{A* Search}
\label{alg1}
\begin{algorithmic}
\STATE exploredSet = $\emptyset$ 
\STATE frontier = [initialPath]
\WHILE{number(explored) $<$ NMAX}
\IF{frontier == $\emptyset$}
    \RETURN FALSE
\ENDIF
\STATE path = frontier.pop()
\STATE state = path[0]
\STATE exploredSet.add(state)
\IF{state == goalState}
    \RETURN path
\ENDIF
\FOR{action in state.validActions()}
    \FOR{newState in action.results()}
        \STATE newPath = path + newState
        \IF{ismember(frontier,newPath) == FALSE}
            \STATE frontier.push(newPath)
        \ENDIF
    \ENDFOR
\ENDFOR
\ENDWHILE
\end{algorithmic}
\end{algorithm}

For implementing the frontier, we use the priority queue, which is actually a heap data structure. We use a callback function $f(state) = g(state) + h(state)$ as the priority, where $g(state)$ is the length of the path and $h(state)$ is the heuristic for estimating the distance between the current state to the goal state. We will analyze several admissble and non-admissble heuristics in section \ref{sec:exp}.


\section{Recursive Best-First Search}

Recursive best-first search or RBFS works by storing an f-limit for each node. The algorithm uses the f-limit to decide which subtree of the problem tree to explore by considering the best and 2nd best (alternative) f-limits. In order to keep the search functional, RBFS also updates the f-values of each node during the search.

The advantage of RBFS over A* is that RBFS uses less memory. Whereas A* stores all of its explored nodes, RBFS will only keep relevant nodes in memory. However, the disadvantage of RBFS over A* is that RBFS could expand more nodes than A* due to redundancy. Since RBFS does not store all nodes explored, it can re-expand the same nodes and thereby increasing computation time.

The pseudocode is listed in figure \ref{alg2}.

\begin{algorithm}
\caption{RBFS Search}
\label{alg2}
\begin{algorithmic}
\STATE function RBFS(state, f-limit):
\IF{state is goal state}
	\RETURN solution
\ENDIF
\STATE successor = all children of state
\IF{successors is empty}
	\RETURN failure
\ELSE
	\FOR{s in successors}
		\STATE s.$f$ = $max$(s.$g$ + s.$h$, state.$f$)
	\ENDFOR
	\WHILE{\TRUE}
		\STATE best = lowest f-value node in successors
		\IF{best.$f$ $>$ f-limit}
			\RETURN failure, best.$f$
		\ENDIF
		\STATE alternative = second best f-value of any node in successors
		\STATE result, best.$f$ = RBFS(best, $min$(f-limit, alternative))
		\IF{result is not failure}
			\RETURN result
		\ENDIF
	\ENDWHILE
\ENDIF
\end{algorithmic}
\end{algorithm}

We will analyze several admissble and non-admissble heuristics in section \ref{sec:exp}.

%\begin{table}[h]
%    \centering
%    \begin{tabular}{|l|l|l|c|c|c|c|c|}
%        \hline
%        WALL & DIRT & HOME & FORWARD & RIGHT & LEFT & SUCK & OFF  \\ \hline
%        1    & 0    & 1    & 0.0     & 0.65  & 0.33 & 0.0  & 0.02 \\ 
%        1    & 1    & 0    & 0.0     & 0.0   & 0.0  & 1.0  & 0.0  \\ 
%        1    & 1    & 1    & 0.0     & 0.0   & 0.0  & 1.0  & 0.0  \\ 
%        1    & 0    & 0    & 0.0     & 0.67  & 0.33 & 0.0  & 0.0  \\ 
%        0    & 0    & 1    & 0.9     & 0.05  & 0.05 & 0.0  & 0.0  \\ 
%        0    & 1    & 0    & 0.0     & 0.0   & 0.0  & 1.0  & 0.0  \\ 
%        0    & 1    & 1    & 0.0     & 0.0   & 0.0  & 1.0  & 0.0  \\ 
%        0    & 0    & 0    & 0.8     & 0.1   & 0.1  & 0.0  & 0.0  \\
%        \hline
%    \end{tabular}
%    \caption{Parameters of multinomial distributions for each situation.}\label{tab:random}
%\end{table}

\section{Experiments}\label{sec:exp}

\begin{table}[!h]
    \centering
    \scalebox{0.8}{
	    \begin{tabular}{|l|l|c|c|c|c|c|c|c|c|c|c|}
		\hline
		1 & 2 & 3 & 4 & 5 & 6 & 7 & 8 & 9 & 10 & 11 & 12 \\ \hline
		1 & 2 & 3 & 4 & 5 & 6 & 7 & 8 & 9 & 10 & 11 & 12 \\ \hline
		1 & 2 & 3 & 4 & 5 & 6 & 7 & 8 & 9 & 10 & 11 & 12 \\ \hline
		1 & 2 & 3 & 4 & 5 & 6 & 7 & 8 & 9 & 10 & 11 & 12 \\ \hline
		1 & 2 & 3 & 4 & 5 & 6 & 7 & 8 & 9 & 10 & 11 & 12 \\ \hline
        \hline
		1 & 2 & 3 & 4 & 5 & 6 & 7 & 8 & 9 & 10 & 11 & 12 \\ \hline
		1 & 2 & 3 & 4 & 5 & 6 & 7 & 8 & 9 & 10 & 11 & 12 \\ \hline
		1 & 2 & 3 & 4 & 5 & 6 & 7 & 8 & 9 & 10 & 11 & 12 \\ \hline
		1 & 2 & 3 & 4 & 5 & 6 & 7 & 8 & 9 & 10 & 11 & 12 \\ \hline
		1 & 2 & 3 & 4 & 5 & 6 & 7 & 8 & 9 & 10 & 11 & 12 \\ \hline
        \hline
		1 & 2 & 3 & 4 & 5 & 6 & 7 & 8 & 9 & 10 & 11 & 12 \\ \hline
		1 & 2 & 3 & 4 & 5 & 6 & 7 & 8 & 9 & 10 & 11 & 12 \\ \hline
		1 & 2 & 3 & 4 & 5 & 6 & 7 & 8 & 9 & 10 & 11 & 12 \\ \hline
		1 & 2 & 3 & 4 & 5 & 6 & 7 & 8 & 9 & 10 & 11 & 12 \\ \hline
		1 & 2 & 3 & 4 & 5 & 6 & 7 & 8 & 9 & 10 & 11 & 12 \\ \hline
        \hline
		1 & 2 & 3 & 4 & 5 & 6 & 7 & 8 & 9 & 10 & 11 & 12 \\ \hline
		1 & 2 & 3 & 4 & 5 & 6 & 7 & 8 & 9 & 10 & 11 & 12 \\ \hline
		1 & 2 & 3 & 4 & 5 & 6 & 7 & 8 & 9 & 10 & 11 & 12 \\ \hline
		1 & 2 & 3 & 4 & 5 & 6 & 7 & 8 & 9 & 10 & 11 & 12 \\ \hline
		1 & 2 & 3 & 4 & 5 & 6 & 7 & 8 & 9 & 10 & 11 & 12 \\ \hline
        \hline
		1 & 2 & 3 & 4 & 5 & 6 & 7 & 8 & 9 & 10 & 11 & 12 \\ \hline
		1 & 2 & 3 & 4 & 5 & 6 & 7 & 8 & 9 & 10 & 11 & 12 \\ \hline
		1 & 2 & 3 & 4 & 5 & 6 & 7 & 8 & 9 & 10 & 11 & 12 \\ \hline
		1 & 2 & 3 & 4 & 5 & 6 & 7 & 8 & 9 & 10 & 11 & 12 \\ \hline
		1 & 2 & 3 & 4 & 5 & 6 & 7 & 8 & 9 & 10 & 11 & 12 \\ \hline
        \hline
		1 & 2 & 3 & 4 & 5 & 6 & 7 & 8 & 9 & 10 & 11 & 12 \\ \hline
		1 & 2 & 3 & 4 & 5 & 6 & 7 & 8 & 9 & 10 & 11 & 12 \\ \hline
		1 & 2 & 3 & 4 & 5 & 6 & 7 & 8 & 9 & 10 & 11 & 12 \\ \hline
		1 & 2 & 3 & 4 & 5 & 6 & 7 & 8 & 9 & 10 & 11 & 12 \\ \hline
		1 & 2 & 3 & 4 & 5 & 6 & 7 & 8 & 9 & 10 & 11 & 12 \\ \hline
        \hline
		1 & 2 & 3 & 4 & 5 & 6 & 7 & 8 & 9 & 10 & 11 & 12 \\ \hline
		1 & 2 & 3 & 4 & 5 & 6 & 7 & 8 & 9 & 10 & 11 & 12 \\ \hline
		1 & 2 & 3 & 4 & 5 & 6 & 7 & 8 & 9 & 10 & 11 & 12 \\ \hline
		1 & 2 & 3 & 4 & 5 & 6 & 7 & 8 & 9 & 10 & 11 & 12 \\ \hline
		1 & 2 & 3 & 4 & 5 & 6 & 7 & 8 & 9 & 10 & 11 & 12 \\ \hline
        \hline
		1 & 2 & 3 & 4 & 5 & 6 & 7 & 8 & 9 & 10 & 11 & 12 \\ \hline
		1 & 2 & 3 & 4 & 5 & 6 & 7 & 8 & 9 & 10 & 11 & 12 \\ \hline
		1 & 2 & 3 & 4 & 5 & 6 & 7 & 8 & 9 & 10 & 11 & 12 \\ \hline
		1 & 2 & 3 & 4 & 5 & 6 & 7 & 8 & 9 & 10 & 11 & 12 \\ \hline
		1 & 2 & 3 & 4 & 5 & 6 & 7 & 8 & 9 & 10 & 11 & 12 \\ \hline
	    \end{tabular}
    }
    \caption{Caption here.}\label{tab:results}
\end{table}

\section{Discussion}\label{sec:dis}

We place some further discussion points in this section.

It appears that rule two was not as effective in reducing the domains of variables. Perhaps this is due to the effectiveness of rule one, since rule one appears a lot and updates have to be propagated around the board anyway. It may also be the order in which we applied our rules since it was applied after rule one.

The naked doubles and triple rules were effective, eliminating backtracking entirely for most problems. This means in real life, the user did not have to make an educated guess for a slot, which is the assignment step of backtracking search. Another possibility is that the user made all successful guesses when he had to; the educated guesses were successful thanks to the most constrained heuristic.

This makes sense because a SuDoKu problem should be solvable with just constraint propagation and a set of rules, no matter how fancy they are, even if the rules are more sophisticated than the naked triples. However, a puzzle creator can easily amuse (or annoy) a solver by forcing him to do backtracking search, which translates to "guessing" squares. Backtracking search forces the user to make guess assignments in real life, and if they make an assignment that contradicts, they have to "erase" their progress and try again by backtracking. We anticipated the use of backtracking search in harder problems, which was correct.

Thefore, we evaluate the following conjecture: "easy problems may be solved by only using the first rule, medium problems may be solved by using the first two rules, and hard and evil problems require the naked triples rule and possibly backtracking." It seems that the conjecture is almost true. Almost all of the easy problems were solvable by the first rule (21 out of 23). Once naked doubles and triples were added, all easy problems were solved, most medium problems were now solved (19 out of 21) and almost all hard and evil problems were solved. It appears that rule two did nothing to help medium problems or any problems at all. However, adding backtracking to any problem solves the problem. These trends are observable in table \ref{tab:dif}. 

Finally, each puzzle was solvable on the order of a second, even with the random slot heuristic. This heavily contrasts the Towers of Corvallis puzzle that used just search, which solved on the order of minutes to hours. While we would need to perform experiments to formalize this heavy contrast, intuitively it shows that constraint propagation takes advantage of the factored representation of each state to outperform general search.


\bibliographystyle{plain}
\bibliography{ref}

\end{document}
This is never printed
